\documentclass[11pt,openany,a4paper]{article}

%import package
\usepackage[utf8]{inputenc}
\usepackage[T1]{fontenc}
\usepackage{lmodern}
\usepackage[french]{babel}
\usepackage{amsfonts,amsmath,amssymb,amsthm}
\usepackage{geometry} %fix the margin of the document
\geometry{top=3cm, bottom=3cm, left=2.5cm, right=2.5cm}
\usepackage{xargs} %define new command
\usepackage{graphicx} %insertion image
\usepackage{caption} %insertion de légendes et titres
\usepackage{indentfirst}
\usepackage{color}
\usepackage[table]{xcolor}
\usepackage{float}
\usepackage{tcolorbox}
\usepackage{appendix} % Appendices

% Bibliographie
\usepackage[backend=biber,sorting=nyt,citestyle=authoryear,bibstyle=alphabetic]{biblatex}% Bibliographie
\addbibresource{biblio.bib} 
\usepackage{csquotes}
\usepackage{url}

%title shape %\MakeUppercase
\usepackage{titlesec}
\titleformat{\chapter}[display]
{\normalfont\huge\bfseries\center}{\chaptertitlename\ \thechapter}{20pt}{\Huge}
\titleformat{\section}
{\normalfont\Large\bfseries\center}{\thesection}{1em}{}
\titleformat{\subsection}
{\normalfont\large\center}{\thesubsection}{1em}{}
\titleformat{\subsubsection}
{\normalfont\normalsize\bfseries\center}{\thesubsubsection}{1em}{}
\titleformat{\paragraph}[runin]
{\normalfont\normalsize\bfseries}{\theparagraph}{1em}{}
\titleformat{\subparagraph}[runin]
{\normalfont\normalsize\bfseries}{\thesubparagraph}{1em}{}

%news commands
\newcommand{\f}[2]{\frac{#1}{#2}}
\newcommand{\lp}{\left(}
\newcommand{\rp}{\right)}
\newcommand{\lb}{\left|}
\newcommand{\rb}{\right|}
\newcommand{\lc}{\left[}
\newcommand{\rcc}{\right]}
\newcommand{\la}{\left\langle}
\newcommand{\ra}{\right\rangle}
\newcommand{\dd}{\;\mathrm{d}}
\newcommand{\ddt}[1]{\frac{\partial #1}{\partial t}}
\newcommand{\R}{\mathbb{R}}
\newcommand{\C}{\mathbb{C}}
\newcommand{\Z}{\mathbb{Z}}
\newcommand{\N}{\mathbb{N}}
\newcommand{\spec}{\operatorname{spec}}
\newcommand{\specp}{\operatorname{spec_p}}
\newcommand{\essran}[1]{\operatorname{ess_{#1}ran}}
\newcommand{\esssup}[1]{\operatorname{ess_{#1}sup}}
\newcommand{\rd}{\mathbb{R}^d}

\newcommand{\St}[2]{e^{-i #1 #2}}
\newcommand{\Stt}[2]{e^{i #1 #2}}
\newcommand{\ortho}{P^\perp}
\newcommand{\Ker}{\operatorname{Ker}}
\newcommand{\im}{\operatorname{Im}}
\newcommand{\pvm}{\mathrm{d}\langle \rangle}
\newcommand{\suminf}[2]{\sum_{#1=#2}^{+\infty}}

%new sections
\newtheorem{Def}{Définition}
\newtheorem{defprop}{Définition/Proposition}
\newtheorem{prop}{Proposition}
\newtheorem{theo}{Théorème}
\newtheorem{lem}{Lemme}
\theoremstyle{definition}
\newtheorem{ass}{Hypothèses}
\newtheorem{cor}{Corollaire}
\theoremstyle{definition}
\newtheorem{Not}{Notation}
\theoremstyle{definition}
\newtheorem{ex}{Exemple}
\theoremstyle{definition}
\newtheorem{rem}{Remarque}


\everymath{\displaystyle}

\title{The Fermi Golden Rule}
\author{Théo Duez }

\begin{document}

\maketitle

\tableofcontents

\newpage

\section{Introduction}

\subsection{La mécanique quantique}

\subsection{La règle d'or de Fermi} 

Qu'est ce que la règle d'or de Fermi ? A ce propos, Wikipédia [] dit efficacement ceci : \\

\begin{it}
    "En physique quantique, la règle d'or de Fermi est un moyen de calculer le taux de transition (probabilité de transition par unité de temps) à partir d'un état propre énergétique d'un système quantique vers un continuum d'états propres, par perturbation."
\end{it}\\

Cette définition mérite un certains nombres d'explications que nous allons  prendre le temps de donner. 

\paragraph{Spectre continue}
Comme nous l'avons rappelé, tout système quantique est régit par un opérateur linéaire, appelé Hamiltonien et noté $H$, auto-adjoint. Cet opérateur peut agir soit sur des espaces  de dimension fini, dans ce cas $H$ est simplement une matrice, ou des espaces de dimension infinies, à tire d'exemple citons le Laplacien agissant sur $H^2(\R^3)$. Les grandeurs qui peuvent être mesurées sur ce systèmes sont les valeurs propres de cet opérateur, le système se projetant alors sur le sous-espace propre associé. S'il est aisé d'étudier le spectre de matrice, étudier le spectre d'opérateur de dimension infini est plus délicat et se traire dans des cours de Théorie Spectrale. Nous renvoyons en annexe à quelques définitions et grands principes de la théorie spectrale et supposerons que le lecteur a connaissance des notions abordés dans l'annexe. Il est important de noter que la théorie spectrale en dimension infinie est certes plus général que celle de la dimension finie, mais implique nombre de résultats et de notions plus délicates qui sont complètement absents du monde fini-dimensionnel et qui ne peuvent donc pas être "intuité". Illustrons cela.
- spectre opérateur compact, opérateurs continues -> cela vient du fait que différence injective <=> bijective <=> surhective
L'hamiltonien $H$ possède la propriété d'être auto-adjoint. Il s'agit là d'une généralisation d'être symétrique pour des matrices réelles ou hermitiennes pour des matrices complexes. Il est possible de montrer qu'il est possible de daignoliser, dans un sens généraliser, tout opérateurs auto-adjoints, dont les valeurs propres sont alors réelles. Nous renvoyons encore ici à l'annexe pour plus de détails. Cette hypothèse qui vient des physiciens est assea naturel si l'on cherche à faire les postulats [] et [].

\paragraph{Pertubation}
Expliquons maintenant ce que l'on entend par perturbation en phsyique quantitique. Il arrive souvent que le système que l'on étudie est soumise à une "petite" perturbation, c'est à dire que notre hamiltonien de départ se retrouve remplacé par un nouvel hamitonien $H(\epsilon) := H_0 + \epsilon H_1$ où $H_1$ est aussi un opérateur autoadjoint représentant la perturbation et $\epsilon$ un réel positif qui a vocation a être petit.  Par exemple, cela peut modéliser []. En présence de tels perturbations, les valeurs propres initiales peuvent se retouver pertuber, et la connaissance des valeurs propres étant fondamentales en phsyiqye quantique, des physiciens ont développé des théories pour calculer, étudier, caractérisé les valeurs propres perturbées et qui ont été reprises rigoureusement par des mathématiciens comme la théorie de Kato qui stipule de nombreux résultats selon la régularité des domaines des opérateurs $H_0$ et $H_1$. 

\paragraph{Pertubation}


En résumé, 

\subsection{Objectifs de ce projet}

Dire les notations et les hypothèses que l'on fait

\section{Approche par les résonances}

Dans cette partie, nous présentons l'approche dite par les résonances, dénomination que nous expliquerons dans la suite. Commençons par fixer les notations qui seront valables dans toute la suite de cette partie. Soit $H$ l'opérateur Hamiltonien 


\subsection{The Livsic Matrix}

Comme expliquer plus haut, une des approches pour établir la règle d'or de Fermi consiste à se ramener à étudier la quantité impliquant la résolvante : $\la \phi_0  | (H-z)^{-1} \phi_0\ra$ où $\phi_0$ appartient à un sous-espace de $\mathcal{H}$ fini dimensionnel noté $E_0$. Si on note $P$ le projecteur orthogonal sur ce sous-espace, on a que $P\phi_0 = \phi_0$ et donc $\la \phi_0  | e^{-Ht} \phi_0\ra = \la P\phi_0  | (H-z)^{-1} P\phi_0\ra = \la \phi_0  | P(H-z)^{-1} P\phi_0\ra$ car tout projecteur orthogonal est auto-adjoint (). Si $E_0$ est de dimension 1, on a même que $\langle \phi_0  | = P$ et $|\phi_0  \rangle = P^* = P$. On voit donc que pour établir la règle d'or de Fermi, il convient d'étudier l'opérateur $P(H-z)^{-1}P$.

\begin{prop}[Livsic Matrix]
    Soit $H$ un opérateur autoadjoint sur un espace de Hilbert.
\end{prop}

\begin{rem}
Cette notion de matrice de Livsic fait écho avec sa version finie dimensionnelle bien connue : le complément de Schur.  
Soient $n,m \in\N^*$. Considérons une matrice $A\in\C^{(n+m)\times (n+m)}$ hermitienne s'écrivant 
\begin{equation*}
    A:= \begin{pmatrix}
A_{11} & A_{12} \\
A_{21} & A_{22} 
\end{pmatrix}
\end{equation*}
avec $A_{11}\in\R^{n\times n},  A_{12}\in\C^{n\times m}, A_{21}\in\C^{m\times n}$ et $A_{22}\in\C^{m\times m}$. Soit $\lambda \in \R$ et cherchons à résoudre le problème aux valeurs propres $Ax = \lambda x$ d'inconnue $x\in\R^{n+m}$ en supposant que $\lambda$ ne soit pas valeur propre de $A_{22}$. La deuxième ligne de ce système conduit à l'égalité $x_2 = -(A_{22}-\lambda)^{-1}A_{21}x_1$ que l'on peut injecter dans la première ligne et obtenir $(A_{11} - A_{12}(A_{22}-\lambda)^{-1}A_{21})x_1 = \lambda x_1 $. On voit que l'on a bien réduit la dimension du problème de départ et que $\lambda$ est valeur propre de $A$ si et seulement si $\lambda$ est valeur propre de $(A_{11} - A_{12}(A_{22}-\lambda)^{-1}A_{21})$.
\end{rem}

\subsection{Howland's Theorem}

Dans son article intitulé \textit{The Livsic  Matrix in Perturbation Theory}, James S. Howland propose en 1975 un théorème de concentration spectrale intéressant, sous l'hypothèse relativement forte mais qui est souvent vérifiée : la possibilité d'effectuer un prolongement analytique de la matrice de Livisc du plan complexe supérieur vers le plan complexe inférieur.

\begin{theo}[\textbf{Concentration spectrale - Howland (1975)}]
Soit $(H_n)$ une suite d'opérateur auto-adjoints convergent fortement dans le sens généralisé vers un opérateur autoadjoint $H$. Soit $\lambda_0$ une valeur propre de $H$ de multiplicité finie et soit $E_0 = \Ker(H -\lambda_0)$. Soient $B_n(z)$  et $B(z) = \lambda_0 I_m$ les matrices de Livsic sur $E_0$ pour $H_n$ et $H$ respectivement. 
Supposons que \begin{enumerate}
    \item[(1)] pour tout $n\in\N$, $B_n(z)$ possède un prolongement analytique $B_n^+(z)$ du plan complexe supérieur sur un voisinage $ \Omega$ de $\lambda_0$,
    \item[(2)] $\lp B_n^+(z)\rp$ converge fortement vers $B(z)$ uniformément sur $\Omega$.
\end{enumerate}
Alors, pour $n$ assez grand, l'équation sur $z$ : $\det(B_n^+(z) -z) = 0 $ possède exactement $m$ solutions comptées avec multiplicités dans $\Omega$. De plus, si on note $\xi_k(n) = \lambda_k(n) - i \Gamma_k(n)$ pour $k\in\{1,\cdots,m\}$ ces solutions, et si on choisis une suite $(\delta_k(n))$ de réels positifs de sorte que $\Gamma_k(n) = o(\delta_k(n))$, alors en définissant les intervalles 
$$ J_k(n) = \bigcup_{k=1}^m (\lambda_k(n) - \delta_k(n),\lambda_k(n) + \delta_k(n))$$ alors $P = st-\lim E_n(J_n)$.
\end{theo}

\begin{proof}
Commençons par prouver l'existence des zéros. Par hypothèse, on a que la suite de matrices $(B_n(z))$ converge uniformément sur $\Omega$ vers $B(z)$, ce qui implique, en combinant au fait que le déterminant est continue, que pour $n$ assez grand et tout $z\in\Omega$, $$ |\det (B_n^+(z) - z) - \det (B(z) - z)| < \epsilon = |\det (B(z) - z)| $$ car $|B_n^+(z) - z - (B(z) - z)| \leq ||B_n - B|| \rightarrow 0$. Ceci étant vrai pour tout $z\in\Omega$, ceci est vrai pour tout $z\in\gamma$ où $\gamma$ est un chemin inclu dans $\Omega$ et entourant $\lambda_0$. Comme $\lambda_0$ est l'unique zéro et de multiplicité $m$ de $\det (B(z) - z)$, par théorème de Rouché [], l'équation $\det(B_n^+(z) -z) = 0$ a exactement $m$ solutions comptées avec multiplicité.

\end{proof}

\newpage

\begin{proof}
En utilisant successivement la formule de Cauchy, l'égalité définissant les matrices de Livsic, 
\begin{eqnarray*}
\la \varphi_0,e^{-iH(\epsilon)t}\varphi_0 \ra - \la\varphi_0,e^{-iH_n(\epsilon)t}\varphi_0 \ra &=& \f{1}{2i\pi}\oint_\Gamma e^{-itz}\la \varphi_0,\lp H(\epsilon) -z)^{-1} - (H_n(\epsilon)-z)^{-1} \rp \varphi_0 \ra dz\\
&=& \f{1}{2i\pi}\oint_\Gamma e^{-itz} \lp B(z,\epsilon) -z)^{-1} - (B_n(z,\epsilon)-z)^{-1} \rp dz
\end{eqnarray*}
\end{proof}


\subsection{Orth's Theorem}

\section{Second Approach : Davies's Theory}





\newpage

\appendix
\appendixpage
\addappheadtotoc
\section{Introduction of the main Mathematical tools}

Pour étudier la règle d'or de Fermi, et les problèmes de mécaniques quantiques en général, les mathématiciens ont recours à une théorie fondamentale : la théorie spectrale. Cette théorie a pour but d'étudier le comportement d'opérateurs linéaires sur des espaces de Hilbert ou de Banach potentiellement (et surtout ) de dimension infinie, et de leurs spectres, généralisant les études bien connues sur les matrices, opérateur de dimension finie. 

[Continuer intro]

Il faut un cours entier de M2 pour établir les éléments introductifs de la théorie spectrale et commencer à entrevoir les théorie sous-jacente, comme la théorie des perturbations de Kato, le principe du maximum ou la théorie de la diffusion. Cette théorie requiert 

\subsection{Premières définitions} (1 Pages)

- Déf opérateur, bornée, non-bornée, symétrique
- Déf adjoint d'un opérateur, déf autoadjoint (généralisation de Hermitienne)
- Exemples

\subsection{Spectre et Résolvant} (1.5 pages)

- Déf résolvant, spectre et différents types de spectre
- Identité du résolvant
- Opérateur conjugué => même spectre
- Spectre opérateur compact, auto-adjoint (juste propriétés), quelques inégalité et petit principe du maximum


\subsection{Calculs fonctionnels} (1.5 pages)

Intro pour dire pourquoi on s'intéresse à ca : e(iH), généralisation du spectre d'un opérateur T
- Déf convergence forte généralisée

\subsection{Schrodinger-equations}

\section{Biblio}

- Antoine Levitt : Schur Complement
- Stephane Nonnenmacher : COurs de théorie Spectrale
- Wikipédia %https://fr.wikipedia.org/wiki/R%C3%A8gle_d%27or_de_Fermi

\end{document}